\documentclass[10pt,a4paper]{article}
\usepackage[english]{babel}
\usepackage{a4wide}
\usepackage{indentfirst}%缩进
\usepackage{enumerate}%加序号
\usepackage{multirow}%合并行
\usepackage{geometry}%设置页边距
\usepackage[UTF8, scheme = plain]{ctex}
\geometry{top=1.5cm,bottom=1.5cm,left=1.5cm,right=1.5cm}
\begin{document}
\thispagestyle{empty}
\noindent\textbf{\Huge{潘崇聃}}\hfill 生日:1997.9.24\\
学校:上海交通大学 (SJTU)\hfill 地址:上海市闵行区东川路800号(邮编200240)
\\手机:+86 13621631412 \hfill 邮箱: panchongdan@foxmail.com\\\\
\noindent{\textbf{\Large{教育经历}}\\
\rule[0.25mm]{175mm}{0.25mm}\\
\textbf{上海交通大学密西根学院}\hfill\textbf{学士学位\qquad}2016年9月至2020年8月\\
主修:电子与计算机工程
\par -数据结构与算法,计算机组成原理,离散数学,机器学习的优化,概率论的工程应用,线性代数,
\par 现代物理,半导体设备,电磁学,激光,信号和系统,控制系统分析和设计,电子电路等\\
辅修:企业创业
\par -商业管理,企业家的商业基础,供应链管理,技术创业,企业内部创业,国际政治经济关系等\\\\
\noindent{\textbf{\Large{技能}}\\
\rule[0.25mm]{175mm}{0.25mm}\\
\textbf{专业技能:} C/C++, Python, Verilog, Mathematica, Matlab, Solidworks, \LaTeX.\\
\textbf{语言} 普通话(母语),英语(六级618, 托福103, GRE语文152,数学170)}.\\
\textbf{业余技能:} Photoshop, HTML, XML, SQL, Autodesk Inventor, 3D打印.\\\\
\noindent{\textbf{\Large{学术经历}}\\
\rule[0.25mm]{175mm}{0.25mm}\\
\textbf{上海交通大学VEX机器人赛队}\hfill 2018年4月至今 
\par -根据VEX机器人比赛题目,使用Solidworks设计制作18台不同的机器人并参赛,能实现自由小球的高效收集和连续15分钟的稳定发射,用机械臂进行不规则叠装体的抓取和安置,堆叠大量立方体等功能。
\par -使用C语言在机器人的电机和机械结构和传感器上实现高精度的PID控制,滤波等控制算法,使以4000RPM转动的电机的转速误差控制在1.5\%内,并使机器人在10m*10m矩形范围内的运动误差小于5mm\\
\textbf{医用远景呈现机器人}\hfill2019年9月至今  
\par -基于树莓派和3D打印技术开发一台低成本的医用远景呈现机器人。该机器人的功能包括自由移动,远程遥控和监控,药物分发,生命体征监测以及远程视频娱乐等功能,旨在向人们提供远程医疗和保健服务。机器人的远程控制功能主要由树莓派上搭建的gpio网页控制实现。
\par -根据医用远景呈现机器人的开发过程和应用前景,投稿联合论文\emph{Telepresence Robots for Healthcare Management: COVID-19 Experience}至\emph{JMIR}期刊(正在审阅中)。
\par -撰写关于机器人在新冠疫情下应用,缺陷和改进的文献综述,并以此为基础,投稿联合论文\emph{Technology Entrepreneurship in Belt and Road Region: Telepresence Robots for COVID-19}至\emph{IEEE EMR}期刊(正在审阅中)。\\
\textbf{计算机和机器学习项目}\hfill2019年5月至今
\par -使用Verilog语言在FPGA上实现单周期和管线CPU,并输入MIPS指令进行模拟。
\par -使用Xgboost,Sklearn库中的SVM,和Pytorch库中的resnet构建模型,根据华为算法精英大赛中提供的12维数据集预测手机用户的年龄段分布。
\par -分析和优化SVM模型,使其能够对高维空间下耦合的数据进行分类。\\\\
\noindent{\textbf{\Large{实习经历}}\\
\rule[0.25mm]{175mm}{0.25mm}\\
\textbf{上海国际设计创新研究院}\hfill 2019年3月至2019年6月
\par -开发一台可以自由移动,并能自动测量垃圾体积和质量的垃圾桶用于垃圾分类。
\par -建立手机和垃圾桶间的蓝牙通信,并提供控制界面。
\par -通过深度优先搜索算法实现机器人在室内的导航。机器人能通过传感器以网格化记录5m*5m房间中的障碍分布,并实现避障和返回出发点等功能。\\
\textbf{SMG第一财经公益基金会}\hfill 2018年1月至2018年8月
\par -运营和更新基金会网站和公众号
\par -组织公益课程和活动并提供后勤支持。\\\\
\noindent{\textbf{\Large{荣誉和奖学金}}\\
\rule[0.25mm]{175mm}{0.25mm}\\
-\textbf{上海交通大学VEX机器人赛队}队长兼\textbf{交大密西根学院机器人社社长}\\
-\textbf{2019VEX机器人世界锦标赛}技能挑战赛世界冠军,联赛分区冠军,世界亚军。\\
-\textbf{2019VEX机器人亚洲公开赛}全能奖,联赛冠军,技能挑战赛亚军,华硕未来之星。\\
-\textbf{2019VEX机器人亚太锦标赛}联赛冠军,惊奇奖。\\
-\textbf{2019VEX机器人中国锦标赛}联赛亚军,技能挑战赛季军。\\
-\textbf{2019上海交通大学年度人物}提名奖。\\
-\textbf{2020上海市优秀毕业生}。\\
-\textbf{2020中美创客大赛上海赛区}优胜奖。(远程医护机器人)\\
-\textbf{2020中美创客大赛交大校内赛}三等奖。(远程医护机器人)
\end{document}
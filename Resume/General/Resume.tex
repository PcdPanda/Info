\documentclass[10pt,a4paper]{article}
\usepackage[english]{babel}
\usepackage{a4wide}
\usepackage{indentfirst}%缩进
\usepackage{enumerate}%加序号
\usepackage{enumitem}
\usepackage{multirow}%合并行
\usepackage{geometry}%设置页边距
\usepackage{hyperref}
\geometry{top=0.75in,bottom=0.75in,left=0.75in,right=0.75in}
\begin{document}
\begin{center}
    \Large{\textbf{Chongdan Pan}}\\
    \normalsize{pandapcd@umich.edu $\bullet$ 734-934-4277 $\bullet$ Ann Arbor, MI}\\
    \normalsize{\href{https://www.linkedin.com/in/pandapcd/}{www.linkedin.com/in/pcdpanda/}}
\end{center}
\noindent
\rlap{\rule[-1mm]{\linewidth}{.5mm}}\textbf{\large{EDUCATION}}\\
\noindent\textbf{University of Michigan}\hfill Ann Arbor, MI\\
\emph{Master of Science in Information} (GPA: 4/4)\hfill\emph{Apr 2023}\\
Courses: NLP, Data Mining, Time Series Analysis, Database, Front-end Dev\\\\
\textbf{University of Michigan - Shanghai Jiao Tong University Joint Institute}\hfill Shanghai, China\\
\emph{Bachelor of Science in Electrical and Computer Engineering; Minor in Entrepreneurship}\hfill\emph{Aug 2020}\\
Courses: Algorithm, Operating System, Computer Network, Convex Optimization, Machine Learning\\\\
\noindent
\rlap{\rule[-1mm]{\linewidth}{.5mm}}\textbf{\large{SKILLS}}\\
\textbf{Programming: }C/C++, Python, Shell, SQL, Scala, Git, Solidity, Rust, JavaScript, TypeScript\\
\textbf{Third-party tools: }Redis, MongoDB, Hadoop, Alluxio, Spark, Kafka, RabbitMQ, Tableau, Clickhouse\\
\textbf{Framework/Library: }Hadoop, Numpy, Hadoop, Spark, Alluxio, Pytorch, Kafka, RabbitMQ, Django, Hardhat, Anchor\\\\
\noindent
\rlap{\rule[-1mm]{\linewidth}{.5mm}}\textbf{\large{PROFESSIONAL EXPERIENCE}}\\
\textbf{Solana Lab}\hfill Ann Arbor, Michigan\\
Web3 Developer Intern\hfill\emph{June 2022 - Aug 2022}
\begin{itemize}[noitemsep,topsep=0pt]
    \item Led a team to design and built a NFT marketplace with Solana Metaplex API in Rust and TypeScript
    \item Did NFT economy model research to realize NFT unique features with Solana smart contracts.
\end{itemize}
\noindent\\
\textbf{Jump Trading}\hfill Chicago, Illinois\\
Software Engineer Intern\hfill\emph{June 2022 - Aug 2022}
\begin{itemize}[noitemsep,topsep=0pt]
    \item Implemented asynchronous data scanner of the network file system and reduced the latency by three times.
    \item Built user-friendly and high performance Python library to fetch global markets' bar data. 
    \item Developed multidimensional data manipulation and calculation tools for quantitative researchers.
\end{itemize}
\noindent\\
\textbf{Probquant Investment}\hfill Shanghai, China\\
Full-time Software Engineer\hfill\emph{Aug 2020 - Aug 2021}
\begin{itemize}[noitemsep,topsep=0pt]
    \item Used lock-free queue and multithreading to develop a cache-friendly asynchronous logger to reduce the backtest and execution latency of high-frequency trading strategies.
    \item Used Protobuf, ZeroMQ, and RabbitMQ to develop a real-time data exchange and strategy monitoring system between colocated servers.
    \item Built the firm-wide market data warehouse from scratch, enabling quantitative researchers to fetch and save high-frequency trading data with a throughput higher than 10GB/s.
    \item Utilized shared memory to build a stand-alone in-memory key-value database customized for large data matrix, providing 10x higher performance than Redis.
    \item Deployed and tested a distributed RDMA cluster interconnected by InfiniBand.
\end{itemize}
\noindent\\
\rlap{\rule[-1mm]{\linewidth}{.5mm}}\textbf{\large{RESEARCH}}\\
\textbf{Political Misinformation Detection}\hfill \emph{Feb 2022 - Sep 2022}
\begin{itemize}[noitemsep,topsep=0pt]
\item Built a pipeline to scrap and political fack check articles and fetch related suspicious tweets.
\item Applied network and graph methods to detect suspicious misinformation spreaders on Twitter.
\item Used NLP and time series peak method to provide signals for the mass spreading of misinformation.\\
\end{itemize}
\textbf{Entrepreneurship Research on Telepresence Robot}\hfill \emph{Apr 2021 - Aug 2020}
\begin{itemize}[noitemsep,topsep=0pt]
\item Analyzed the telepresence robot's role in Covid-19 based on experiments and literature and built a remote care robot prototype with functions including free movement, medicine dispensation, and vital monitoring.
\item Published a paper \href{https://ieeexplore.ieee.org/document/9330532}{\emph{Technology Entrepreneurship in Developing Countries: Role of Telepresence Robots in Healthcare}} in \emph{IEEE Engineering Management Review}
\item Did a systematic review on the application of telepresence robots and published a paper (preprint) \emph{Telepresence Robots to Support Telehealth during Pandemic} in \emph{Digital Medicine}
\item Authored a book chapter in the IET Book \href{https://shop.theiet.org/digital-tools-and-methods-to-support-healthy-ageing}{\emph{Digital Methods and Tools for Healthy Ageing}}
\item Led a team of 5 members to beat competitors from all over the University and won the champion of \emph{UM-SJTU JI Covid-19 Entrepreneur Challenge}.\\
\end{itemize}
\textbf{Blockchain-based Peer Review System}\hfill\emph{Feb 2022 - April 2022}
\begin{itemize}[noitemsep,topsep=0pt]
    \item Used Solidity to develop an Ethereum-based decentralized application with smart contracts to record the historical behavior of the reviewee for customers' reference.
    \item Authored a book chapter in \emph{Technology Innovations for Disaster Management} to be published by \emph{World Scientific Publishers} based on the idea of using Blockchain to prevent healthcare disasters.
    \item Led an interdisciplinary team of engineers, designers, and analysts to build the front-end and back-end of the project and got into the finalist of Umich Ross Crypto Fintech Challenge.
\end{itemize}
\noindent\\
\rlap{\rule[-1mm]{\linewidth}{.5mm}}\textbf{\large{PROJECT EXPERIENCE}}\\
\textbf{Crypto Market Making}\hfill \emph{May 2021 - present}
\begin{itemize}[noitemsep,topsep=0pt]
    \item Built market data publisher, fair value calculator, and gateways for multiple crypto exchanges to catch arbitrage opportunities.
    \item Used Xetra Liquidity Measure and spread analysis to improve the execution of market maker strategy.
    \item Developed hedge strategy to manage the risk of market making.
\end{itemize}
\textbf{Crypto Quantitive Research}\hfill \emph{May 2021 - present}
\begin{itemize}[noitemsep,topsep=0pt]
    \item Used Word2Vec and LSTM on tweets data to generate meaningful predictors for crypto trading strategies.
    \item Built a data pipeline to fetch, clean, normalize and store median-frequency blockchain market and text data from Binance and Twitter Rest API.
    \item Developed backtest system to test the performance of predictors for Barra models parallelly.
    \item Applied Garch-AR, Bretó Stochastic model to analyze the volatility of crypto market time series data.
\end{itemize}
\textbf{Relic NFT}\hfill\emph{May 2022 - present}
\begin{itemize}[noitemsep,topsep=0pt]
    \item Working with UMich to provide NFT tickets for the match associated with the university sports team.
    \item Used Hardhat, Solidity and Javascript to develop an NFT marketplace supported for trades and standard functions for ERC-721 tokens.
    \item Set up and maintained a forked Ethereum blockchain in our virtual environment to test and run the Dapp.
\end{itemize}

\end{document}
\documentclass[10pt,a4paper]{article}
\usepackage[english]{babel}
\usepackage{a4wide}
\usepackage{indentfirst}%缩进
\usepackage{enumerate}%加序号
\usepackage{enumitem}
\usepackage{multirow}%合并行
\usepackage{geometry}%设置页边距
\usepackage{hyperref}
\geometry{top=0.75in,bottom=0.75in,left=0.75in,right=0.75in}
\begin{document}
\begin{center}
    \Large{\textbf{Chongdan Pan}}\\
    \normalsize{pandapcd@umich.edu $\bullet$ 734-934-4277 $\bullet$ Ann Arbor, MI}\\
    \normalsize{\href{https://www.linkedin.com/in/pcdpanda/}{www.linkedin.com/in/pcdpanda/}}
\end{center}
\noindent
\rlap{\rule[-1mm]{\linewidth}{.5mm}}\textbf{\large{EDUCATION}}\\
\noindent\textbf{University of Michigan}\hfill Ann Arbor, MI\\
\emph{Master of Science in Information} (GPA: 4/4)\hfill\emph{Apr 2023}\\
Topics: NLP, Data Mining, Time Series Analysis, Database, Data Visualization, Web Application\\\\
\textbf{University of Michigan - Shanghai Jiao Tong University Joint Institute}\hfill Shanghai, China\\
\emph{Bachelor of Science in Electrical and Computer Engineering; Minor in Entrepreneurship}\hfill\emph{Aug 2020}\\
Topics: Computer System, Network, Algorithm, Machine Learning, Covex Optimization\\\\
\noindent
\rlap{\rule[-1mm]{\linewidth}{.5mm}}\textbf{\large{SKILLS}}\\
\textbf{Programming: }C/C++, Python, R, Shell, SQL, JavaScript, Java, Scala, Solidity, React Native, HTML/CSS\\
\textbf{Database: }MySQL, Redis, ClickHouse, MongoDB, Firebase\qquad\textbf{Tools: }Git, Jenkins, Docker\\
\textbf{Framework/Library: }Hadoop, Spark, Alluxio, Kafka, RabbitMQ, Scikit-Learn, Pytorch, Django, Altair\\\\
\noindent
\rlap{\rule[-1mm]{\linewidth}{.5mm}}\textbf{\large{EMPLOYMENT EXPERIENCE}}\\
\textbf{Jump Trading}\hfill Chicago, Illinois\\
Software Engineer Intern\hfill\emph{June 2022 - Aug 2022}
\begin{itemize}[noitemsep,topsep=0pt]
    \item Applied concurrent and asynchronous programming to improve the performance of data fetching API.
    \item Designed user-friendly and high performance API to get global markets' bar data. 
    \item Developed multidimensional data manipulation tools for quant researchers.
\end{itemize}
\noindent\\
\textbf{Probquant Investment}\hfill Shanghai, China\\
Full-time Software Engineer\hfill\emph{Aug 2020 - Aug 2021}
\begin{itemize}[noitemsep,topsep=0pt]
    \item Designed a high-performance asynchronous trading logger through a lockfree queue and multithreading.
    \item Used Protobuf, message queues to develop a real-time data exchange system between colocated servers.
    \item Built the company-wide data warehouse from scratch, enabling quantitative researchers to fetch and save high-frequency trading data with throughput higher than 10GB/s.
    \item Utilized shared memory to build a stand-alone in-memory key-value database customized for large data matrix, providing 10x higher performance than Redis.
    \item Deployed and tested a distributed RDMA cluster interconnected by InfiniBand.
\end{itemize}
\noindent\\
\rlap{\rule[-1mm]{\linewidth}{.5mm}}\textbf{\large{PROJECT EXPERIENCE}}\\
\textbf{Crypto Quantitive Research}\hfill \emph{May 2021 - present}
\begin{itemize}[noitemsep,topsep=0pt]
    \item Applied Xetra Liquidity Measure on orderbooks to classify the liquidity curves of different equities.
    \item Used Word2Vec and LSTM on tweets data to generate meaningful predictors for crypto trading strategies.
    \item Built a data pipeline to fetch, clean, normalize and store median-frequency blockchain market and text data from Binance and Twitter Rest API.
    \item Developed backtest system to test the performance of predictors from Barra models parallelly.
    \item Applied Garch-AR, Bretó Stochastic model to analyze the volatility of crypto market time series data.
\end{itemize}
\noindent\\
\textbf{Blockchain-based Peer Review System}\hfill\emph{Feb 2022 - April 2022}
\begin{itemize}[noitemsep,topsep=0pt]
    \item Led an interdisciplinary team of engineers, designers, and analysts to get into the finalist of Umich Ross Crypto Fintech Challenge.
    \item Developed a DApp to record the historical behavior of the reviewee for customers' reference.
    \item Use Solidity to design the smart contracts supporting user interaction and historical data search.
\end{itemize}
\noindent\\
\textbf{President of SJTU VEX Robot Club}\hfill\emph{Sep 2019 - Aug 2020}
\begin{itemize}[noitemsep,topsep=0pt]
    \item \href{https://www.youtube.com/watch?v=7ysdbzLXAT4}{Led an 8-member team to win World Champion in the \emph{2019 VEX World Championship}}.
    \item Implemented control algorithm and signal processing system upon multiple sensors so that the absolute error of 100-liter robots's movement is under 1cm, and the relative error of velocity is under 2\%.
    \item \href{https://www.robotevents.com/teams/VEXU/SJTU1}{Used mechanic designing and fabrication techniques to elaborate more than 20 robots for 3 seasons.}
    \item Founded a club of over 40 members and instructed them to participate in the VEX Robotic Competitions.
\end{itemize}
\end{document}